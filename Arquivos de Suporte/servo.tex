O controle servo-visual é uma técnica que usa o retorno de dados extraídos de um sensor de visão para controlar o movimento do robô. Esse tipo de controle, para o domínio sugerido, se restringe ao controle dos dois servo-mecanismos referentes à cabeça do robô, mais especificamente do pescoço.

Uma das primeiras abordagens referentes ao controle servo-visual foi publicado por \citeonline{Corke}.Trabalhos mais recentes foram publicados por \citeonline{VS1} e por \citeonline{VS2}, aqui de uma forma mais abrangente no qual entende-se por controle servo-visual todas as ações feitas pelo robô que possuem como origem a visão do mesmo, inclusive movimentos das mãos, pés, rodas e articulações.

Técnicas de controle servo visual são classificados nos seguintes tipos:
\begin{enumerate}
     \item Baseada em Imagem (IBVS)
     \item Baseado Posição / Pose  (PBVS)
     \item Abordagem Híbrida
\end{enumerate}

A IBVS foi proposta por \citeonline{Weiss}. A lei de controle é baseada no erro entre as características atuais e desejadas no plano da imagem e não envolve qualquer estimativa da pose do alvo. Os recursos podem ser as coordenadas de recursos visuais, linhas ou momentos de regiões. A IBVS tem dificuldade com movimentos muito grandes, como rotações, o que veio a ser chamado retiro de câmera. \footnote{Do inglês: "Camera Retreat"}

A PBVS é uma técnica baseada em uma única câmera. Isto porque a posição do objeto de interesse é estimada em relação à câmera e, em seguida, é emitido um comando para o controlador de robô, que por sua vez, controla o robô. Neste caso, as características da imagem são extraídas mas, são, também, utilizadas para estimar a pose do objeto no espaço cartesiano.

Abordagens híbridas usam uma combinação dos controles previamente descritos. De forma geral, essas duas técnicas são consideradas passivas já que aguardam os dados do sensor de visão para tomar alguma decisão. 

Uma outra forma de encarar esse desafio é usar uma visão ativa, como proposta por \citeonline{Sharma}, que, de fato, procura ativamente o objeto a ser identificado e para tanto precisa tomar algumas decisões, mesmo que não haja dados relevantes vindos da câmera. 

Um sistema de visão ativo é capaz de interagir com o seu meio ambiente, alterando o seu ponto de vista em vez de observá-lo passivamente e operando em sequências de imagens, em vez de em um único quadro. Além disso, a capacidade de acompanhar fisicamente um alvo reduz o borrão de movimento, aumentando a resolução de destino para tarefas de nível mais elevado, tais como classificação. 

O controle servo-visual armazena as posições de trajetória do objeto e, por meio de regressão, estima as próximas posições do mesmo.  	Usando um algoritmo guloso encontra-se o melhor movimento antecipado da câmera em direção ao objeto. Para garantir um movimento plausível da câmera, é necessário considerar a cinemática e a dinâmica desse movimento, otimizando a trajetória da câmera para que essa possa manter o objeto em seu campo de visão.

Uma das vertentes da visão ativa é a configuração escravo-mestre \cite[p.2]{Fernandez}, no qual uma câmera está fixa e outra está em movimento. Ambas precisam estar em um mesmo sistema de coordenadas para que as informações trocadas entre elas sejam facilmente interpretadas. Essa configuração poderá ser usada em um projeto futuro, no qual o goleiro exerça o papel de câmera estática de supervisão enquanto os outros jogadores seriam as câmeras ativas em movimento. 

