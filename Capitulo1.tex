
\addtocounter{page}{-2} % A capa não é contada para efeito de páginas

Um robô é um sistema computacional com sinais de entrada provenientes dos sensores e um conjunto de atuadores que permitem ao agente interagir com o ambiente ao qual está inserido. O conceito de robôs jogadores de futebol foi introduzida em 1993 no Japão, e até hoje, é uma das mais importantes divisões da RoboCup. O objetivo oficial da organização é apresentar, até 2050, um time de robôs jogadores de futebol totalmente autônomos capazes de vencer uma partida, usando as regras oficiais da FIFA, contra a seleção humana vencedora da copa do mundo. 

Quatro robôs autônomos (figura ~\ref{Fig:Robos}) foram desenvolvidos pelo time humanoide da FEI para atuar na liga KidSize da RoboCup, de maneira que cada um pudesse apresentar somente características humanas, ou seja, precisavam ter o formato humanoide. O dispositivo de aquisição de imagens precisa estar posicionada na cabeça, além de, necessitar de um processo de decisão e de visão individual livre de controles externos e, é claro, nenhum sensor, a não ser que represente um sentido humano. 


\section{Uma Breve Introdução à Visão Computacional de Robôs Móveis Humanoides}
\label{Intro}
A robótica móvel foca em robôs que possuem a capacidade de se mover em um determinado ambiente. Os robôs móveis precisam interagir com o ambiente, muitas vezes complexo, e isso por si só não é trivial. As dificuldades decorrem de como esse agente artificial sente e manipula esse ambiente.  Robôs humanoides foram criados devido ao desejo do ser humano pela criação de um robô que fosse capaz de realizar as atividades feitas pelo homem, e que também possuísse uma semelhança física, mas também, para sobressair as limitações que os robôs sobre rodas possuem.

\begin{figure}[!t]
\centering \caption{Robôs desenvolvidos.}
\includegraphics[width=12cm]{Imagens/robos.jpg}
\DeclareGraphicsExtensions.
\caption*{Fonte: Autor}
\label{Fig:Robos}
\end{figure}

A câmera, sendo um importante sensor do ambiente, proporciona imagens cheias de informação que o robô precisa processar e, a partir dai, tomar uma decisão. Os robôs humanoides atuais possuem uma câmera com é o caso do Darwin-OP \cite{Darwin} ou duas câmeras que podem estar posicionadas de maneira estéreo como é o caso do robô do time Nubots \cite{Nubots} \cite{Edrom}, ou duas câmeras posicionadas de maneira a uma voltada para o horizonte e outra voltada 45 graus na direção do chão, como é o caso do \citeonline{NAO}. 




\subsection{Tipos de Sistemas de Visão Computacional para Robôs Móveis}

Dois tipos de sistemas de visão computacional são normalmente utilizados o sistema de visão passiva e o sistema de visão ativa. O sistema de visão computacional passiva para robôs móveis se vale do conceito reativo de agir \cite{Tsotsos}. \citeonline{Tsotsos} ao se referir ao sistema de visão passiva menciona a pouca complexidade da busca pelo objeto de forma passiva reagindo ao seu movimento. \citeonline{Aloimonos} introduziram o primeiro formato geral para um sistema de visão ativa no final de 1980 a fim de melhorar a qualidade de percepção de rastreamento de objetos procurando-os e estimando sua posição futura. O sistema de visão ativa é particularmente importante para lidar com problemas como oclusões, campo de visão limitado e resolução da câmera limitada \cite{Denzler}.  Outras vantagens são: Reduzir o borrão causado por um objeto em movimento \cite{Rivlin}, reforçar a percepção de profundidade de um objeto concentrando-se duas câmeras no mesmo objeto, ou movimentar as câmeras \cite{Aloimonos} com o mesmo fim. Finalmente, câmeras autônomas são câmeras que podem direcionar a si mesmas no ambiente. \citeonline{Denzler} usaram um conjunto de câmeras estéreis para acompanhar o movimento de um objeto.

\section{Motivação}
\label{Mot}
A motivação do presente trabalho é desenvolver um sistema de visão computacional robusto e rápido para um robô humanoide, já que um dos maiores problemas da visão computacional é ter que lidar com a quantidade de informação, muitas vezes irrelevante, e algoritmos que requerem grande 
esforço de processamento. Ser capaz de identificar quais os pixels de uma imagem compõem um objeto é um problema não-trivial. Os trabalhos publicados para o domínio proposto se baseiam em operações que lidam diretamente com todos os pixels da imagem, ditas operações de baixo e médio nível, geralmente com baixo custo computacional. De modo que o desenvolvimento desse sistema complemente essas técnicas com o uso de técnicas de alto nível, ou seja com alguma inferência de contexto ao agrupamento de pixels.


\section{Objetivos}
\label{Obj}
Este projeto visa desenvolver um sistema servo-visual para um robô humanoide atuante na competição RoboCup, Liga humanoide, categoria KidSize. Serão desenvolvidos sistemas de visão computacional que permitam ao agente:

\begin{enumerate}
	\item rastrear uma bola, de tamanho conhecido, fornecendo as coordenadas de sua posição;
	\item identificar agentes do mesmo time e do time rival;
\end{enumerate}

\section{Justificativas}
\label{Jus}
O desenvolvimento de um time de futebol de robôs representa uma aplicação do uso de agentes autônomos com comportamento inteligente, visando a execução de uma tarefa, geralmente complexa, em equipe. O estudo e desenvolvimento da inteligência artificial, fornece desafios e problemas onde diversas tecnologias e metodologias podem ser combinadas para a obtenção dos resultados desejados, especialmente em visão computacional.



\section{Histórico}
\label{His}
A RoboCup é uma iniciativa educacional de pesquisa, é também uma competição cujo objetivo é incentivar a pesquisa robótica fornecendo desafios nos quais uma grande quantidade de tecnologias possam ser integradas. A cada ano uma nova cidade é escolhida em um pais diferente para sediar esse evento. Em 1992, o professor Alan Mackworth, em suas próprias palavras, considerou a possibilidade de robôs jogarem futebol:

\begin{quote}
``Suppose we want to build a robot to play soccer. Quite apart from all the difficult robotics and perception problems involved, we have substantial challenges in representation for planning and action'' \\ \cite{Mackworth}
\end{quote} 

Na tradução livre do inglês:  'Suponha que quiséssemos construir um robô para jogar futebol. Além de, todos os problemas robóticos e de percepção envolvidos, teremos desafios substanciais em representações do planejamento e das ações'. Após essa menção, os pesquisadores: Minoru Asada, Yasuo Kuniyoshi, e Hiroaki Kitano organizaram um workshop em Tóquio, fato que culminou, um ano mais tarde, em uma competição de robótica limitada apenas às fronteiras nipônicas chamada Robot J-League. Em 1997, essa competição tornou-se mundial e teve seu nome alterado para RoboCup. 

O objetivo e meta da RoboCup, foi bem sintetizada na frase da própria RoboCup Federation:
\begin{quote}
``Em meados de século 21, uma equipe totalmente autônoma de robôs humanoides jogadores de futebol deve vencer um jogo contra o time de humanos campeão da última Copa do Mundo da FIFA, utilizando as regras da FIFA.'' \cite{siteobjetivoRoboCup}. 
\end{quote}

Essa frase propõe que este objetivo será um dos grandes desafios compartilhados pelas comunidades de robótica e de Inteligência Artificial para os próximos 50 anos. A RoboCup, por possuir um intuito científico, possui além das diversas competições, o simpósio que acontece logo após as mesmas. Nele os participantes, podem publicar seus trabalhos científicos relevantes da área.

\section{Domínio}
\label{Dom}
O projeto tem como base a principal competição de futebol para robôs, a RoboCup, sendo a liga de atuação do agente a liga Humanoide na categoria KidSize. Todos os dados dessa seção estão de acordo com o livro de regras de 2014 da RoboCup. \cite{Rules}

O campo de jogo consiste de uma superfície plana e nivelada, coberta por um carpete verde liso na regras de 2014 \cite{Rules} e com grama artificial nas regras de 2015. Segmentos de linha de 10 centímetros de largura são usados para a marca de penalty e a posição inicial do início da partida. O campo é limitado pelas linha maiores, chamadas laterais, e por linhas menores, definidas como sendo de fundo, onde se encontram os gols. Todas as áreas localizadas fora dessas quatro linhas são consideradas desconhecidas e indefinidas. Os gols são compostos por duas traves, um travessão e uma rede. As condições de iluminação dependem do local da competição. Em volta do campo, um local é definido onde somente o juiz, seus assistentes e mais dois manipuladores dos robôs podem atuar. Nenhum participante pode usar, abaixo da cintura, cores iguais ou similares às definidas para o campo, bola ou gols. 

Apenas os sensores equivalentes aos sentidos humanos são permitidos, mas, não são obrigatórios. Robôs humanoides devem ter o formato de seus corpos semelhante ao do ser humano, devem ter pernas e braços. Uma série de restrições são aplicadas às medidas dos agentes, uma delas, que tem mais relevância para a pesquisa, é a relação entre a altura do corpo do robô \(H_{\text{Robô}}\) e a altura da cabeça \(H_{\text{Cabeça}}\), de forma que, a altura da cabeça, incluindo o pescoço, deve satisfazer a seguinte expressão \cite{Rules}:\hfill 

%\hspace{80}
\begin{equation}
0,05 * H_{\text{Robô}} \leq H_{\text{Cabeça}} \leq 0,25 * H_{\text{Robô}}  \\
\end{equation}
Onde a altura da cabeça é definida, como sendo, a distância do eixo da primeira junta do braço, no ombro, até o topo da cabeça.

\begin{table}[ht!]
    \caption{Medidas Importantes do Campo, da Bola e dos Jogadores} \label{tbl:medidas}
    \centering
    \begin{tabular}{|c|c|c|c|}
    \hline 
    Descrição & Comprimento (cm) \\ 
    \hline 
    Linha Lateral ( Comprimento do Campo ) & 900  \\ 
    \hline 
    Linha de Fundo ( Largura do Campo ) & 600 \\ 
    \hline 
    Diâmetro do grande círculo & 150 \\ 
    \hline 
    Diâmetro da Bola & 10 \\ 
    \hline 
    Altura Máxima dos Agentes & 90 \\ 
    \hline 

    \end{tabular}
    \caption*{Fonte: Livro de Regras RoboCup - (RoboCup Rules Book), 2014 \cite{Rules}}
\end{table}

O campo de visão dos robôs é limitado a 180 graus, isso significa que o ângulo máximo entre quaisquer dois pontos na sobreposição do campo de visão de todas as câmeras montadas sobre o robô precisam, obrigatoriamente, ser inferiores a 180 graus. O mecanismo para girar (pan) a câmera é limitado a 270 graus, 135 graus para cada lado, contados da posição de olhar para frente. Já o mecanismo para erguer (tilt) a câmera é limitado a 90 graus medidos da linha horizontal, essas restrições simulam as mesmas limitações das juntas do pescoço humano. Seguindo essa linha de limitações humanas, fica definido que de qualquer posição do robô e de qualquer ângulo de câmera ou do robô, o mesmo não pode ser capaz de visualizar os dois gols ao mesmo tempo. O número de câmeras fica limitado a duas, porém, a visão monocular também é permitida, de qualquer forma deve(m) estar na cabeça do robô.

Robôs participantes das competições da liga humanoide devem ter corpos, predominantemente, pretos, cinzas, prateados ou brancos, com as limitações de serem não reflexivos e terem o pés pretos. Qualquer cor do campo, gol, bola ou time adversário deve ser evitada no robô. Braços, pernas e corpos devem ter aparência sólida e devem estar marcadas com a cor magenta (para agentes do time vermelho) e com a cor ciano (para agentes do time azul), ambas as marcas devem ter no mínimo 5 centímetros e estar visíveis nos dois lados das pernas e braços.

\section{Organização do Trabalho}

As seções anteriores apresentaram uma breve introdução sobre alguns tipos de visão para robôs móveis, a motivação, os principais objetivos, as justificativas, um breve histórico da RoboCup, e finalmente, o domínio ao qual este trabalho esta inserido, compondo o capítulo introdutório. Os demais capítulos estão estruturados da seguinte maneira:

Conceitos básicos de aquisição, processamento e representação de imagens digitais são abordados no capítulo 2, apresentando, em forma de revisão, alguns dos mais conceituados algoritmos de processamento digital de imagens contextualizando-os com a visão computacional. A pesquisa prossegue nesse mesmo capítulo, pondo em perspectiva o modo como algumas equipes participantes da RoboCup solucionaram os mais variados problemas visuais do jogo de futebol. A descrição desse sistema de visão, com fluxogramas e testes a serem feitos, é apresentada na proposta de trabalho abarcada pelo capítulo 3. O capítulo 4 traz os testes já executados com resultados. Por fim, nas Considerações Finais, no capítulo 5, é feita uma breve síntese das atividades realizadas até o momento apresentando as contribuições e apontado para perspectivas futuras.

