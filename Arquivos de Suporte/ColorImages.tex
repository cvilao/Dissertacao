A interpretação de cores feita pelo ser humano é um fenômeno fisiopsicológico, compreendido apenas em partes, porém podemos definir formalmente a natureza da cor experimentalmente e teoricamente. A cor é uma das características mais largamente utilizadas em diversos sistemas por ser relativamente independente quanto ao tamanho, orientação e resolução da imagem e é computacionalmente menos cara quando comparada aos demais descritores \cite{Bender}. 

Para entender como os descritores de cor são utilizados, é necessário definir previamente o conceito de cor e listar algumas de suas características. A cor pode ser definida como um fenômeno perceptual da luz quando incide e é refletida por um objeto. Como se pode observar, o conceito de cor é diretamente relacionado ao conceito de luz \cite {Gonzalez}. Portanto, ao se referir ao conceito de cor deve-se mencionar, obrigatoriamente, a luz.

Da interação entre a energia luminosa e o meio, a cor pode se formar através de (a) processo aditivo, (b) processo subtrativo e (c) formação por pigmentação \cite{Bender}. No processo aditivo ocorre a combinação entre raios de luz com frequências diferentes através da soma de energia dos fótons. No processo de formação subtrativo, a luz é transmitida através de um filtro que absorve a radiação luminosa de um determinado comprimento de onda. Alternativamente, a luz também pode incidir através de um corante, que é constituído por filtros que podem absorver a radiação luminosa de um determinado comprimento de onda. Finalmente, no processo de formação por pigmentação, os pigmentos podem absorver, refletir ou transmitir a radiação luminosa. 

Dois conceitos preliminares são particularmente importantes para o entendimento do conceito de percepção de cor. São eles: a luminância e a crominância. A luminância contém a informação da quantidade das cores pretas e brancas presentes em uma imagem. O cérebro humano interpreta essa informação como a quantidade de cinza presente na cor (ou brilho). A crominância informa a respeito da tonalidade de uma cor. É a frequência dominante do raio de luz. A combinação destes dois conceitos, em diferentes proporções, permite ao cérebro perceber o espectro de cores visível em uma determinada imagem ou cena. Para representar as cores existem modelos ou sistemas de cores \cite {Gonzalez}, como por exemplo RGB (“red, green, blue”), CMY (“cyan, magenta, yellow”),HSI (“matiz, saturação, intensidade”), YIQ (luminância, em-fase, quadratura) e LAB (luminosidade, verde-vermelho, azul-amarelo).

O sistema Red, Green, Blue é um sistema de representação de cor \cite{Bimbo} aditivo e se baseia na teoria dos três estímulos proposta por \cite{Young}. Segundo essa teoria, o olho humano percebe a cor através do estímulo de três pigmentos visuais presentes nos cones da retina, que possuem sensibilidades para alguns comprimentos de onda, como por exemplo, 630 nanômetros (vermelho – red), 530 nanômetros (verde – green) e 450 nanômetros (azul – blue). Uma representação desse sistema pode ser formulada através de um cubo com os eixos R, G e B. Podemos dizer que a origem representa a cor preta, os vértices das coordenadas (1, 1, 1) representam o branco, os vértices que estão sobre os eixos representam as cores primárias e os vértices restantes representam os complementos das primárias. 

É importante observar que cada ponto no interior do cubo corresponde a uma cor, representada por uma tripla (R, G, B), com os valores de R, G e B variando de 0 a 1. Os tons de cinza são representados ao longo da diagonal principal do cubo, que se inicia no ponto de origem até o vértice que foi apontado como a cor branca. Portanto, é possível, afirmar que cada tom de cinza presente no cubo é formado por contribuições iguais das cores primárias que o compõem. Um exemplo de tom de cinza médio poderia ser representado pela tripla (0.5, 0.5, 0.5).

