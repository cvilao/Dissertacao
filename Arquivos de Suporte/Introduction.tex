
A Robocup é uma iniciativa educacional de pesquisa, é também uma competição cujo objetivo é incentivar a pesquisa robótica fornecendo desafios nos quais uma grande quantidade de tecnologias possam ser integradas. Jogadores de xadrez artificiais, como o Deep Blue, são bons exemplos do que esse tipo de iniciativa pode fazer pela robótica e a inteligência artificial. 

O conceito de robôs jogadores de futebol foi introduzida em 1993 no Japão, e até hoje, é uma das mais importantes divisões da Robocup. Em 2014, essa competição foi celebrada no Brasil. O objetivo oficial da organização é apresentar, até 2050, um time de robôs jogadores de futebol totalmente autônomos capazes de vencer uma partida, usando as regras oficiais da FIFA, contra a seleção vencedora da Copa do mundo. 

Quatro robôs autônomos (figura ~\ref{Fig:Robos}) foram desenvolvidos pelo time humanoide da FEI para atuar na liga Kid Size da Robocup, de maneira que cada um pudesse apresentar somente características humanas, ou seja, precisavam ter o formato humanoide. O dispositivo de aquisição de imagens precisa estar posicionada na cabeça, além de, necessitar de um processo de decisão e de visão individual livre de controles externos e, é claro, nenhum sensor, a não ser que represente um sentido humano. 



\section{Motivação}

Onde os algoritmos propostos na literatura não se enquadram no contexto. 

A motivação do presente trabalho é desenvolver um sistema de visão computacional robusto e rápido para um robô humanoide, já que um dos maiores problemas da visão computacional, é ter que lidar com a quantidade de informação, muitas vezes irrelevante, e algoritmos que requerem grande esforço de processamento.  





\section{Objetivos}

Este projeto visa desenvolver um sistema servo-visual para um robô humanoide atuante na competição Robocup no ano de 2014, Liga humanoide, categoria Kid Size. Serão desenvolvidos sistemas de visão computacional que permitam ao agente:\\

\begin{enumerate}
	\item Rastrear uma bola, de tamanho conhecido e cor discrepante do fundo, fornecendo sua posição em um sistema de coordenadas;
	\item Identificar linhas de um campo de futebol, assim, como o gol, de modo que essas informações possam ser utilizadas para o agente tomar uma decisão;
	\item Reconhecer agentes do mesmo time e do time rival, possibilitando a realização de passes entre robôs;
	\item Determinar se ele se encontra orientado para o campo de ataque ou de defesa.

\end{enumerate}

\section{Justificativas}
O desenvolvimento de um time de futebol de robôs representa uma aplicação do uso de agentes autônomos com comportamento inteligente, visando a execução de uma tarefa, geralmente complexa, em equipe. O estudo e desenvolvimento da inteligência artificial, fornece desafios e problemas onde diversas tecnologias e metodologias podem ser combinadas para a obtenção dos resultados desejados, especialmente em visão computacional.

\begin{figure}[!t1]
\centering
\includegraphics[width=5.5in]{Imagens/robos.jpg}
\DeclareGraphicsExtensions.
\caption{Robôs desenvolvidos.\\ Fonte: o Autor}
\label{Fig:Robos}
\end{figure}

\section{Histórico}

A primeira edição da Robocup aconteceu em 1997 em Nagoya, no Japão, depois de se ter organizado uma pré-Robocup em Osaka, no mesmo país, com finalidades de encontrar possíveis dificuldades e problemas na organização desse tipo evento em escala global. A iniciativa foi lançada pelo Dr. Hiroaki Kitano (Tóquio), um pesquisador na área de Inteligência Artificial que se tornou presidente e fundador da Robocup Federation.


\begin{table}[ht!]
    \caption{Histórico Robocup} 
    \centering
    \begin{tabular}{|c|c|c|c|}
    \hline 
    Competição & Ano & Cidade & País Sede \\ 
    \hline 
    Robocup & 2015 & Hefei & China \\ 
    \hline 
    Robocup & 2014 & João Pessoa & Brasil \\ 
    \hline 
    Robocup & 2013 & Eindhoven & Holanda \\ 
    \hline 
    Robocup & 2012 & Cidade do México & México \\ 
    \hline 
    Robocup & 2011 & Istambul & Turquia \\ 
    \hline 
    Robocup & 2010 & * & Singapura \\ 
    \hline 
    Robocup & 2009 & Graz & Áustria \\ 
    \hline 
    Robocup & 2008 & Suzhou & China \\ 
    \hline 
    Robocup & 2007 & Atlanta & Estados Unidos \\ 
    \hline 
    Robocup & 2006 & Bremen & Alemanha \\ 
    \hline 
    Robocup & 2005 & Osaka & Japão \\ 
    \hline 
    Robocup & 2004 & Lisboa & Portugal \\ 
    \hline 
    Robocup & 2003 & Pádova & Itália \\ 
    \hline 
    Robocup & 2002 & Fukuoka/Busan & Japão/ Coréia do Sul \\ 
    \hline 
    Robocup & 2001 & Seatle & Estados Unidos \\ 
    \hline 
    Robocup & 2000 & Melbourne & Austrália \\ 
    \hline 
    Robocup & 1999 & Estocolmo & Suécia \\ 
    \hline 
    Robocup & 1998 & Paris & França \\ 
    \hline 
    Robocup & 1997 & Nagoya & Japão \\ 
    \hline 
    \end{tabular}
	\label{tbl:historico}
    \caption*{Fonte: \citeonline{Rules}}
\end{table}

\section{Domínio}

O projeto tem como base a principal competição de futebol para robôs, a Robocup, sendo a liga de atuação do agente a Humanoid League na categoria Kid Size. Todos os dados dessa seção estão de acordo com o livro de regras de 2014 da Robocup.

O campo de jogo consiste de uma superfície plana e nivelada, coberta por um carpete verde. As linhas brancas têm 5 centímetros de largura. Segmentos de linha de 10 centímetros de largura são usados para a marca de penalty e a posição inicial do início da partida. O campo é limitado pelas linha maiores, chamadas laterais, e por linhas menores, definidas como sendo de fundo, onde se encontram os gols. Todas as áreas localizadas fora dessas quatro linhas são consideradas desconhecidas e indefinidas. Os gols são compostos por duas traves, um travessão e uma rede. O travessão, bem como as traves dos gols são amarelos, o trançado da malha da rede não possui aberturas maiores do que 4 centímetros e sua cor poderá ser cinza ou preta. 

São utilizadas apenas lâmpadas como forma de iluminação. Porém, as condições de iluminação dependem do local da competição. O campo é iluminado de forma constante e com brilho suficiente, sem influência da luz do dia. Em volta do campo, um local é definido onde somente o juiz, seus assistentes e mais dois manipuladores dos robôs podem atuar. Nenhum participante pode usar, abaixo da cintura, cores iguais ou similares às definidas para o campo, bola ou gols. Em 2014, a bola possuía coloração laranja, dois times jogavam por vez e cada time de futebol poderia ter no máximo quatro jogadores. 

Apenas os sensores equivalentes aos sentidos humanos são permitidos, mas, não são obrigatórios. Robôs humanoides devem ter o formato de seus corpos semelhante ao do ser humano, devem ter pernas e braços. Uma série de restrições são aplicadas às medidas dos agentes, uma delas, que tem mais relevância para a pesquisa, é a relação entre a altura do corpo do robô \(H_{\text{Robô}}\) e a altura da cabeça \(H_{\text{Cabeça}}\), como segue:

A altura da cabeça, incluindo o pescoço, deve satisfazer a seguinte expressão:\hfill 

%\hspace{80}
\begin{equation}
0,05 * H_{\text{Robô}} \leq H_{\text{Cabeça}} \leq 0,25 * H_{\text{Robô}}  \\
\end{equation}
Onde a altura da cabeça é definida, como sendo, a distância do eixo da primeira junta do braço, no ombro, até o topo da cabeça.

\begin{table}[ht!]
    \caption{Medidas Importantes do Campo, da Bola e dos Jogadores} \label{tbl:medidas}
    \centering
    \begin{tabular}{|c|c|c|c|}
    \hline 
    Descrição & Comprimento (cm) \\ 
    \hline 
    Linha Lateral ( Comprimento do Campo) & 900  \\ 
    \hline 
    Linha de Fundo ( Largura do Campo) & 600 \\ 
    \hline 
    Largura da grande área & 60 \\ 
    \hline 
    Comprimento da grande área & 345 \\ 
    \hline 
    Diâmetro do grande círculo & 150 \\ 
    \hline 
    Diâmetro da Bola & 10 \\ 
    \hline 
    Altura Máxima dos Agentes & 90 \\ 
    \hline 
    Largura dos Gols & 225 \\ 
    \hline
    Altura dos Gols & 110 \\
    \hline
    Altura das malhas dos Gols & 100 \\
    \hline
    Largura das traves e travessões & 10 \\
    \hline


    \end{tabular}
    \caption*{Fonte: Livro de Regras Robocup - (Robocup Rules Book), 2014 \cite{Rules}}
\end{table}

O campo de visão dos robôs é limitado a 180 graus, isso significa que o ângulo máximo entre quaisquer dois pontos na sobreposição do campo de visão, de todas as câmeras montadas sobre o robô, precisam ser inferiores a esse ângulo. 

O mecanismo para girar (pan) a câmera é limitado a 270 graus, 135 graus para cada lado, contados da posição de olhar para frente. Já o mecanismo para erguer (tilt) a câmera é limitado a 90 graus medidos da linha horizontal, essas restrições simulam as mesmas limitações das juntas do pescoço humano. 

Seguindo essa linha de limitações humanas, fica definido que, de qualquer posição do robô, de qualquer ângulo de câmera ou do robô, o mesmo não pode ser capaz de visualizar os dois gols ao mesmo tempo, e o número de câmeras fica limitado a duas, porém, a visão monocular também é permitida, de qualquer forma deve(m) estar na cabeça do robô.

Robôs participantes das competições da liga humanoide devem ser, predominantemente, pretos, cinzas, prateados ou brancos, com as limitações de serem não reflexivos e terem o pés pretos. Qualquer cor do campo, gol, bola ou time adversário deve ser evitada no robô. Braços, pernas e corpos devem ter aparência sólida e devem estar marcadas com a cor magenta (para agentes do time vermelho) e com a cor ciano (para agentes do time azul), ambas as marcas devem ter no mínimo 5 centímetros e estar visíveis nos dois lados das pernas e braços.

\section{Organização do Trabalho}

As seções anteriores apresentaram domínio, justificativas, objetivos, histórico e motivação deste trabalho, compondo o capítulo introdutório. Os demais capítulos estão estruturados da seguinte maneira:

Conceitos básicos de aquisição, processamento e representação de imagens digitais são abordados no capítulo 2, apresentando, em forma de revisão, alguns dos mais conceituados algoritmos de processamento digital de imagens contextualizando-os com a visão computacional, a pesquisa prossegue com o capítulo 3, em que trabalhos correlatos são estudados e é posto em perspectiva o modo como algumas equipes participantes da Robocup utilizam algoritmos de processamento digital de imagens na forma de visão computacional, de maneira que, é possível visualizar como solucionaram os mais variados problemas visuais do jogo de futebol.

A visão do agente foi baseada no que outras equipes têm feito, mas, também, em conceituados algoritmos de visão computacional. A descrição desse sistema de visão, com fluxogramas e testes a serem feitos, é apresentada na proposta de trabalho abarcada pelo capítulo 4. O capítulo 5 traz os testes já executados, os quais geraram alguns resultados preliminares. Por fim, nas Considerações Finais, no capítulo 6, é feita uma breve síntese das atividades realizadas até o momento, apontado para perspectivas futuras, por meio de um Cronograma de atividades.

