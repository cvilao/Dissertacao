\chapter{Algoritmo de Detecção da Bola Laranja} 
\label{chap:apendiceBolaLaranja}
\section{Algoritmo de rastreamento da bola - Círculos de Hough.}

\begin{algorithm}

\Entrada{Abre Captura da Câmera (Dispositivo 0)}
\Se{Captura vazia}
	{Erro! Dispositivo não está pronto.
	Aborte Algoritmo.
		}

\Enqto{ Não for pressionada a tecla esc e Quadro não vazio}
{

Extraia um quadro RGB da captura, converta para uma imagem HSV.

\begin{equation}
     \centering
    	H = \left\{
		\begin{array}
		{ccl} 
		$$\theta, 	& Se B  \leq  G $$\\
		$$360 - \theta,	& Se B  >  G $$
		\end{array} 
\right.
\end{equation}

\begin{equation}
	\theta  =  \cos^-{^1} \left[  {   \frac {  \frac {1} {2} * [(R -G) + (R-B)]}  {  \sqrt{[(R -G)^2 + (R -B)(G -B)]}   }    }\right]
\end{equation}

\begin{equation}
	S  =  1 - { \frac {3} {(R+G+B)} } * [min (R,G,B)]
\end{equation}

\begin{equation}
	I  =  { \frac {1} {3}} * {(R+G+B)}
\end{equation}



\Entrada{Tabela de pesquisa (LUT). (0<H<35, 0<S<180, 0<V<255)}

\ParaTodo {Pixel dentro da imagem HSV e dentro dos limites da LUT}
	{Altere o valor do Pixel de uma matriz binária na mesma posição}

Na matriz binária aplique Erosão seguida da Dilatação. ( 5x5 Pixels)

Aplique filtro espacial. (Gaussiana 9x9 Pixels)

Transformada de Hough para círculos:

- Atualize a média móvel do raio (MRB) e da posição XY dos círculos encontrados.

\begin{equation} 
	D_{bola} = \frac {9143} {MRB}
\end{equation}\\

Armazene \(D_{bola}\), a posição dos servos Pan-Tilt em graus e da bola em pixels.
}

\Retorna \(D_{bola}\)

\caption[Algoritmo de rastreamento da bola - Círculos de Hough.]{Fonte: Autor.}
\label{lst:algBall}
\end{algorithm}


\chapter{Algoritmo para divisão do Quadro} \label{chap:apendiceDivQuadro}

\section{Algoritmo para buscar e centralizar o objeto.}
\begin{algorithm}

Divida o quadro de captura em 3 regiões horizontais, sendo a região central com 5\% da resolução horizontal.\\
Divida o quadro de captura em 3 regiões verticais, sendo a região central com 5\% da resolução vertical.\\
\Enqto{Quadro não vazio}
{
	\Se{Dentro do quadro houver objeto}
		{
		Encontre em qual região o objeto se encontra usando sua posição.\\
				\Se {O objeto estiver na região 5 (central)} 
					{Mantenha o objeto nessa região parando o movimento}
				\Se {O objeto estiver nas regiões 1, 4 e 7} 
					{Some a posição atual do servo com a quantidade de graus para o pan}
				\Se {O objeto estiver nas regiões 3, 6 e 9} 
					{Subtraia a posição atual do servo com a quantidade de graus para o pan}
				\Se {O objeto estiver nas regiões 1, 2 e 3} 
					{Some a posição atual do servo com a quantidade de graus para o tilt}
				\Se {O objeto estiver nas regiões 7, 8 e 9} 
					{Subtraia a posição atual do servo com a quantidade de graus para o tilt}
		Guarde posição dos servos\\
		}
		\Senao
		{		
		Chame algoritmo de busca \ varredura
		}
	
}
\Retorna \(Posicao do Servo\)

\caption[Algoritmo para buscar e centralizar o objeto]{Fonte: Autor}
\label{lst:algCent}
\end{algorithm}



\chapter{Algoritmo para Identificação de Robôs. (HOG-SVM)} \label{chap:apendiceHOG}

\section{HOG-SVM Offline}
\begin{algorithm}


\Entrada{Conjunto de imagens positivas - Contém robôs participantes}
\Entrada{Conjunto de imagens negativas - Não contém robôs}

Extraia, do diretório de imagens negativas, todas as imagens que possuam tamanho e extensão compatíveis (jpg,jpeg,png,bmp)

Calcule o histograma de gradientes para todas as imagens.

Armazene o vetor de características (descritores) de cada uma das imagens negativas.

Extraia, do diretório de imagens positivas, todas as imagens que possuam tamanho e extensão compatíveis (jpg,jpeg,png,bmp)

Calcule o histograma de gradientes para todas as imagens desse diretório.

Armazene o vetor de características (descritores) de cada uma das imagens positivas.

Armazene o vetor de características (descritores) de cada uma das imagens negativas.

\Entrada{Vetores de características (descritores) das imagens positivas}
\Entrada{Vetores de características (descritores) das imagens negativas}

Determine os vetores de suporte e o hiperplano que melhor separa as duas classes usando o SVM.

\Entrada{Tabela com todas as alturas de todos os robôs de todos os times}
\Entrada{Tabela com a cor do time adversário}

\label{lst:algROF}
\caption[Algoritmo para Identificação de Robôs. (HOG-SVM Offline)]{Fonte: Autor.}
\end{algorithm}

\pagebreak

\section{HOG-SVM Online}

\begin{algorithm}


\Entrada{Hiperplano proveniente do SVM}
\Entrada{Aquisição da câmera}

\Enqto{ Quadro não vazio}
{
Calcule o histograma de gradientes da imagem adquirida da câmera

Extraia o vetor de caracterísiticas (descritores) da imagem adquirida.

Use o SVM para classificar o vetor de caracterísiticas da imagem adquirida.

	\Se {Houver robô presente na imagem}
	{
		Para cada bloco da imagem adquirida, encontre em qual bloco do histograma normalizado o robô se encontra\\

		Calcule sua altura em pixels\\

		Armazene o centro do bloco e a altura em pixels\\

		Segmentação da cor do time.\\

		Se a área na qual o robô for identificado conter o centroide referente à cor do time\\

		Busque na tabela de alturas qual a altura em centímetros do robô que foi identificado\\

		Calcule sua distância real até o robô em centímetros\\
	}

}
\caption[Algoritmo para Identificação de Robôs. (HOG-SVM Online)]{Fonte: Autor.}
\label{lst:algROn}
\end{algorithm}


\chapter{Algoritmo para Identificação de Robôs. (HAAR-Adaboost)} \label{chap:apendiceHAAR}

\section{HAAR-Adaboost Offline}
\begin{algorithm}

\Entrada{Conjunto de imagens positivas - Contém robôs participantes}
\Entrada{Conjunto de imagens negativas - Não contém robôs}

\Entrada{Posição dos robôs contidos nas imagens positivas}

\ParaTodo {Estágio da árvore e para todas as imagens}
{
\Enqto{\(Min_{\text{TaxaAcerto}}\) = 0,999 não atingida}
	{
	Extraia as características de HAAR para as imagens positivas.\\
	Execute o Adaboost conforme algoritmo ~\ref{lst:algHaar} usando como parâmetros:\\
		- Tamanho da amostra = 20x20 Pixels;\\
		- Tamanho mínimo do objeto = 30x30 Pixels;\\
		- Tamanho máximo do objeto = Resolução vertical x \( \frac {Resolução Horizontal}{2}\) Pixels\\
		- Classificador 8 Vizinhos mais próximos. \\
	}
	Armazene a hipótese final do estágio.\\
}
	Armazene a hipótese final em um arquivo .xml

\caption[Algoritmo de Identificação de robôs. (HAAR-Adaboost Offline)]{Fonte: Autor ``Adaptado de'' \citefloat{Adaboost} }
\label{lst:algHaarOF}
\end{algorithm}

\pagebreak

\section{HAAR-Adaboost Online}
\begin{algorithm}

\Entrada{Árvore proveniente do Adaboost (Arquivo .xml)}

\Entrada{Aquisição da câmera}

\Enqto{ Quadro não vazio}
{

Extraia o vetor de caracterísiticas (descritores) da imagem adquirida.\\
Percorra os nós (Estágios) da árvore presentes no arquivo xml encontrando a melhor classificação para a imagem adquirida\\

	\Se {Houver robô presente na imagem}
	{
		Para cada bloco da imagem adquirida, encontre em qual bloco o robô se encontra\\

		Calcule sua altura em pixels\\

		Armazene o centro do bloco e a altura em pixels\\

		Segmentação da cor do time.\\

		Se a área na qual o robô for identificado conter o centroide referente à cor do time\\

		Busque na tabela de alturas qual a altura em centímetros do robô que foi identificado\\

		Calcule sua distância real até o robô em centímetros\\
	}
}

\caption[Algoritmo de Identificação de robôs. (HAAR-Adaboost Online)]{Fonte: Autor ``Adaptado de''\citefloat{Adaboost} }
\label{lst:algHaarOn}
\end{algorithm}


	
